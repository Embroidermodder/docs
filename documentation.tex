\hypertarget{embroidermodder-1.90.0-manual}{%
\section{Embroidermodder 1.90.0
Manual}\label{embroidermodder-1.90.0-manual}}

\hypertarget{table-of-contents}{%
\subsection{Table of Contents}\label{table-of-contents}}

\begin{enumerate}
\def\labelenumi{\arabic{enumi}.}
\tightlist
\item
  \protect\hyperlink{introduction}{Introduction}
\item
  \protect\hyperlink{basic-features}{Basic Features}
\item
  \protect\hyperlink{advanced-features}{Advanced Features}
\item
  \protect\hyperlink{other-projects}{Other Projects}
\item
  \protect\hyperlink{References}{References}
\end{enumerate}

\hypertarget{introduction}{%
\subsection{Introduction}\label{introduction}}

\hypertarget{basic-features}{%
\subsection{Basic Features}\label{basic-features}}

\hypertarget{move-a-single-stitch-in-an-existing-pattern}{%
\subsubsection{Move a single stitch in an existing
pattern}\label{move-a-single-stitch-in-an-existing-pattern}}

\begin{enumerate}
\def\labelenumi{\arabic{enumi}.}
\tightlist
\item
  In the \texttt{File} menu, click \texttt{Open...}. When the open
  dialog appears find and select your file by double clicking the name
  of the file. Alternatively, left click the file once then click the
  \texttt{Open} button.
\item
\item
  In the \texttt{File} menu
\end{enumerate}

TIP: For users who prefer

\hypertarget{convert-one-pattern-to-another-format}{%
\subsubsection{Convert one pattern to another
format}\label{convert-one-pattern-to-another-format}}

\begin{enumerate}
\def\labelenumi{\arabic{enumi}.}
\tightlist
\item
  In the \texttt{File} menu, click \texttt{Open...}.
\item
  The
\item
  In the dropdown menu within the save dialog select the
\end{enumerate}

\hypertarget{advanced-features}{%
\subsection{Advanced Features}\label{advanced-features}}

\hypertarget{other-projects}{%
\subsection{Other Projects}\label{other-projects}}

\hypertarget{references}{%
\subsection{References}\label{references}}

\hypertarget{ideas}{%
\section{Ideas}\label{ideas}}

\hypertarget{why-this-document}{%
\subsection{Why this document}\label{why-this-document}}

I've been trying to make this document indirectly through the Github
issues page and the website we're building but I think a
straightforward, plain-text file needs to be the ultimate backup for
this. Then I can have a printout while I'm working on the project.

\hypertarget{issues}{%
\subsection{Issues}\label{issues}}

\hypertarget{fix-before-version-2}{%
\subsubsection{Fix before Version 2}\label{fix-before-version-2}}

So I've had a few pieces of web infrastructure fail me recently and I
think it's worth noting. An issue that affects us is an issue that can
effect people who use our software.

\begin{enumerate}
\def\labelenumi{\arabic{enumi}.}
\tightlist
\item
  Googletests require a web connection to update and they update on each
  compilation.
\item
  Downloading and installing Qt has been a pain for some users (46Gb on
  possibly slow connections). I think it was davieboy64?
\item
  The documentation is, well better in that it's housed in the main
  repository, but I'm not a fan of the ``write once build many''
  approach as it means trying to weigh up how 3 versions are going to
  render.
\item
  Github is giving me a server offline (500) error and is still giving a
  bad ping.
\item
  OpenGL rendering within the application. This will allow for Realistic
  Visualization - Bump Mapping/OpenGL/Gradients?
\item
  JSON configuration (Started, see
  \texttt{head\ -n\ 50\ src/mainwindow.cpp.}) Ok this is changing
  slightly. embroidermodder should boot from the command line regardless
  of whether it is or is not installed (this helps with testing and
  running on machines without root). Therefore, it can create an
  initiation file but it won't rely on its existence to boot: this is
  what we currently do with settings.ini.
\item
  Get undo history widget back (BUG).
\item
  Switch website to a CMake build.
\item
  Mac Bundle, .tar.gz and .zip source archive.
\item
  NSIS installer for Windows, Debian package, RPM package
\item
  GUI frontend for embroider features that aren't supported by
  embroidermodder: flag selector from a table
\item
  Update all formats without color to check for edr or rgb files.
\item
  EmbroideryFLOSS - Color picker that displays catalog numbers and names
\item
  Setting for reverse scrolling direction (for zoom, vertical pan)
\item
  Stitching simulation
\item
  User designed custom fill
\item
  Keyboard zooming, panning
\item
  Advanced printing
\item
  Libembroidery 1.0
\item
  Better integrated help: I don't think the help should backend to a
  html file somewhere on the user's system. A better system would be a
  custom widget within the program that's searchable.
\item
  New embroidermodder2.ico 16x16 logo that looks good at that scale.
\item
  saving dst, pes, jef
\item
  Settings dialog: notify when the user is switching tabs that the
  setting has been changed, adding apply button is what would make sense
  for this to happen.
\item
  Update language translations
\item
  Replace KDE4 thumbnailer.
\item
  Import raster image
\item
  Statistics from 1.0, needs histogram.
\item
  SNAP/ORTHO/POLAR
\item
  Cut/copy allow post-selection
\item
  Layout into config
\item
  Notify user of data loss if not saving to an object format.
\item
  Add which formats to work with to preferences.
\item
  Cannot open file with \# in the name when opening multiple files but
  works with opening a single file.
\item
  Closing settings dialog with the X in the window saves settings rather
  than discarding them.
\item
  Otto theme icons: units, render, selectors, what's this icon doesn't
  scale
\item
  Layer manager and Layer switcher dock widget
\item
  test that all formats read data in correct scale (format details
  should match other programs).
\item
  Custom filter bug -- doesn't save changes in some cases.
\end{enumerate}

CAD command review:

\begin{enumerate}
\def\labelenumi{\arabic{enumi}.}
\tightlist
\item
  scale
\item
  circle
\item
  offset
\item
  extend
\item
  trim
\item
  BreakAtPoint
\item
  Break2Points
\item
  Fillet
\item
  star
\item
  singlelinetext
\item
  Chamfer
\item
  split
\item
  area
\item
  time
\item
  pickadd
\item
  zoomfactor
\item
  product
\item
  program
\item
  zoomwindow
\item
  divide
\item
  find
\item
  record
\item
  playback
\item
  rotate
\item
  rgb
\item
  move
\item
  grid
\item
  griphot
\item
  gripcolor
\item
  gripcool
\item
  gripsize
\item
  highlight
\item
  units
\item
  locatepoint
\item
  distance
\item
  arc
\item
  ellipse
\item
  array
\item
  point
\item
  polyline
\item
  polygon
\item
  rectangle
\item
  line
\item
  arc (rt)
\item
  dolphin
\item
  heart
\end{enumerate}

So, it means weighing up some simplifications.

\hypertarget{fix-for-version-2.1}{%
\subsubsection{Fix for Version 2.1}\label{fix-for-version-2.1}}

\hypertarget{fix-eventually}{%
\subsubsection{Fix eventually}\label{fix-eventually}}

\hypertarget{googletests}{%
\subsubsection{googletests}\label{googletests}}

gtests are non-essential, testing is for developers not users so we can
choose our own framework. I think the in-built testing for libembroidery
was good and I want to re-instate it.

\hypertarget{qt-and-dependencies}{%
\subsubsection{Qt and dependencies}\label{qt-and-dependencies}}

I'm switching to SDL2 (which is a whole other conversation) which means
we can ship it with the source code package meaning only a basic build
environment is necessary to build it.

\hypertarget{documentation}{%
\subsubsection{Documentation}\label{documentation}}

Can we treat the website being a duplicate of the docs a non-starter?
I'd be happier with tex/pdf only and (I know this is counter-intuitive)
one per project.

\hypertarget{social-platform}{%
\subsubsection{Social Platform}\label{social-platform}}

So\ldots{} all the issues and project boards etc. being on Github is all
well and good assuming that we have our own copies. But we don't if
Github goes down or some other major player takes over the space and we
have to move (again, since this started on SourceForge).

This file is a backup for that which is why I'm repeating myself between
them.

\hypertarget{json-data-ideas}{%
\subsection{JSON data Ideas}\label{json-data-ideas}}

So:

\begin{enumerate}
\def\labelenumi{\arabic{enumi}.}
\tightlist
\item
  Port \texttt{settings.ini} to \texttt{settings.json}.
\item
  Place \texttt{settings.json} in \texttt{\$HOME/.embroidermodder} (or
  equivalent, see the homedir function in \texttt{gui.c}).
\item
  Parse JSON using cJSON (we have the new parseJSON function).
\item
  Better structure for settings data so parse and load JSON is easier
  and there's less junk in global variables. A structure similar to a
  Python dict that uses constants like the sketch below.
\end{enumerate}

\hypertarget{why-json-over-ini}{%
\subsubsection{Why JSON over ini?}\label{why-json-over-ini}}

\begin{enumerate}
\def\labelenumi{\arabic{enumi}.}
\tightlist
\item
  We need to hand-write \emph{a} system because the current system is Qt
  dependent anyway.
\item
  This way we can store more complex data structures in the same system
  including the layout of the widgets which may be user configured (see
  Blender and GIMP).
\item
  Also it's easier to share information formatted this way between
  systems because most systems us JSON or XML data: there's better
  support for converting complex data this way.
\end{enumerate}

\hypertarget{sketch-of-a-settings-system}{%
\subsubsection{Sketch of a settings
system}\label{sketch-of-a-settings-system}}

\begin{verbatim}
#define SETTING_interface_scale    16

...

char int_settings_labels[] = {
...
"interface scale" /* the sixteenth entry */
...
"%" /* terminator character */
};

...

    /* to use the setting */
    scale_interface(int_setting[SETTING_interface_scale]);

    /* to set setting */
    int_setting[SETTING_interface_scale] = 16;

    /* to make the JSON storage */
    for (i=0; int_settings_labels[i][0] != '%'; i++) {
        fprintf(setting_file, "\"%s\" :%d,\n", int_settings_labels[i], int_settings[i]);
    }
\end{verbatim}

This would all be in C, and wouldn't rely on Qt at all. We already use a
system like this in \texttt{libembroidery} so hopefully devs on both
would get the pattern.

\hypertarget{design}{%
\subsection{Design}\label{design}}

These are key bits of reasoning behind why the software is built the way
it is.

\hypertarget{scripting-overhaul}{%
\subsubsection{Scripting Overhaul}\label{scripting-overhaul}}

Originally Embroidermodder had a terminal widget, this is why we removed
it.

\begin{quote}
ROBIN: I think supporting scripting within Embroidermodder doesn't make
sense.

All features that use scripting can be part of libembroidery instead.
Users who are capable of using scripting won't need it, they can alter
their embroidery files in CSV format, or import pyembroidery to get
access. It makes maintaining the code a lot more complicated, especially
if we move away from Qt. Users who don't want the scripting feature will
likely be confused by it, since we say that's what libembroidery,
embroider and pyembroidery are for.

How about a simpler ``call user shell'' feature? Similar to texmaker we
just call system on a batch or shell script supplied by the user and it
processes the file directly then the software reloads the file. Then we
aren't parsing it directly.

I don't want to change this without Josh's support because it's a fairly
major change.

\begin{quote}
JOSH: I totally agree.

I like the idea of scripting just so people that know how to code could
write their own designs without needing to fully build the app.
Scripting would be a very advanced feature that most users would be
confused by. Libembroidery would be a good fit for advanced features.
\end{quote}
\end{quote}

\hypertarget{perennial-jobs}{%
\subsection{Perennial Jobs}\label{perennial-jobs}}

\begin{enumerate}
\def\labelenumi{\arabic{enumi}.}
\tightlist
\item
  Check for memory leaks
\item
  Clear compiler warnings on \texttt{-Wall\ -ansi\ -pedantic} for C.
\item
\end{enumerate}

\hypertarget{developing-for-android}{%
\subsubsection{Developing for Android}\label{developing-for-android}}

https://developer.android.com/studio/projects/add-native-code

\begin{verbatim}
apt install google-android-ndk-installer cmake lldb gradle
\end{verbatim}

\hypertarget{bibilography}{%
\subsection{Bibilography}\label{bibilography}}

\hypertarget{introduction-1}{%
\subsection{Introduction}\label{introduction-1}}

\hypertarget{basic-features-1}{%
\subsection{Basic Features}\label{basic-features-1}}

\hypertarget{move-a-single-stitch-in-an-existing-pattern-1}{%
\subsubsection{Move a single stitch in an existing
pattern}\label{move-a-single-stitch-in-an-existing-pattern-1}}

\begin{enumerate}
\def\labelenumi{\arabic{enumi}.}
\tightlist
\item
  In the \texttt{File} menu, click \texttt{Open...}. When the open
  dialog appears find and select your file by double clicking the name
  of the file. Alternatively, left click the file once then click the
  \texttt{Open} button.
\item
\item
  In the \texttt{File} menu
\end{enumerate}

TIP: For users who prefer

\hypertarget{convert-one-pattern-to-another-format-1}{%
\subsubsection{Convert one pattern to another
format}\label{convert-one-pattern-to-another-format-1}}

\begin{enumerate}
\def\labelenumi{\arabic{enumi}.}
\tightlist
\item
  In the \texttt{File} menu, click \texttt{Open...}.
\item
  The
\item
  In the dropdown menu within the save dialog select the
\end{enumerate}

\hypertarget{advanced-features-1}{%
\subsection{Advanced Features}\label{advanced-features-1}}

\hypertarget{other-projects-1}{%
\subsection{Other Projects}\label{other-projects-1}}

\hypertarget{references-1}{%
\subsection{References}\label{references-1}}

\hypertarget{planning}{%
\subsection{Planning}\label{planning}}

To see what's planned open the
\href{https://github.com/Embroidermodder/Embroidermodder/projects/1}{Projects}
tab which sorts all of the GitHub Issues into columns.

\hypertarget{format-support}{%
\subsection{Format Support}\label{format-support}}

\begin{longtable}[]{@{}llll@{}}
\toprule
\begin{minipage}[b]{0.19\columnwidth}\raggedright
FORMAT\strut
\end{minipage} & \begin{minipage}[b]{0.22\columnwidth}\raggedright
READ\strut
\end{minipage} & \begin{minipage}[b]{0.25\columnwidth}\raggedright
WRITE\strut
\end{minipage} & \begin{minipage}[b]{0.22\columnwidth}\raggedright
NOTES\strut
\end{minipage}\tabularnewline
\midrule
\endhead
\begin{minipage}[t]{0.19\columnwidth}\raggedright
10o\strut
\end{minipage} & \begin{minipage}[t]{0.22\columnwidth}\raggedright
YES\strut
\end{minipage} & \begin{minipage}[t]{0.25\columnwidth}\raggedright
\strut
\end{minipage} & \begin{minipage}[t]{0.22\columnwidth}\raggedright
read (need to fix external color loading) (maybe find out what
ctrl\strut
\end{minipage}\tabularnewline
\begin{minipage}[t]{0.19\columnwidth}\raggedright
100\strut
\end{minipage} & \begin{minipage}[t]{0.22\columnwidth}\raggedright
\strut
\end{minipage} & \begin{minipage}[t]{0.25\columnwidth}\raggedright
\strut
\end{minipage} & \begin{minipage}[t]{0.22\columnwidth}\raggedright
none (4 byte codes) 61 00 10 09 (type, type2, x, y ?) x \& y (signed
char)\strut
\end{minipage}\tabularnewline
\begin{minipage}[t]{0.19\columnwidth}\raggedright
art\strut
\end{minipage} & \begin{minipage}[t]{0.22\columnwidth}\raggedright
\strut
\end{minipage} & \begin{minipage}[t]{0.25\columnwidth}\raggedright
\strut
\end{minipage} & \begin{minipage}[t]{0.22\columnwidth}\raggedright
none\strut
\end{minipage}\tabularnewline
\begin{minipage}[t]{0.19\columnwidth}\raggedright
bro\strut
\end{minipage} & \begin{minipage}[t]{0.22\columnwidth}\raggedright
YES\strut
\end{minipage} & \begin{minipage}[t]{0.25\columnwidth}\raggedright
\strut
\end{minipage} & \begin{minipage}[t]{0.22\columnwidth}\raggedright
read (complete)(maybe figure out detail of header)\strut
\end{minipage}\tabularnewline
\begin{minipage}[t]{0.19\columnwidth}\raggedright
cnd\strut
\end{minipage} & \begin{minipage}[t]{0.22\columnwidth}\raggedright
\strut
\end{minipage} & \begin{minipage}[t]{0.25\columnwidth}\raggedright
\strut
\end{minipage} & \begin{minipage}[t]{0.22\columnwidth}\raggedright
none\strut
\end{minipage}\tabularnewline
\begin{minipage}[t]{0.19\columnwidth}\raggedright
col\strut
\end{minipage} & \begin{minipage}[t]{0.22\columnwidth}\raggedright
\strut
\end{minipage} & \begin{minipage}[t]{0.25\columnwidth}\raggedright
\strut
\end{minipage} & \begin{minipage}[t]{0.22\columnwidth}\raggedright
(color file no design) read(final) write(final)\strut
\end{minipage}\tabularnewline
\begin{minipage}[t]{0.19\columnwidth}\raggedright
csd\strut
\end{minipage} & \begin{minipage}[t]{0.22\columnwidth}\raggedright
YES\strut
\end{minipage} & \begin{minipage}[t]{0.25\columnwidth}\raggedright
\strut
\end{minipage} & \begin{minipage}[t]{0.22\columnwidth}\raggedright
read (complete)\strut
\end{minipage}\tabularnewline
\begin{minipage}[t]{0.19\columnwidth}\raggedright
dat\strut
\end{minipage} & \begin{minipage}[t]{0.22\columnwidth}\raggedright
\strut
\end{minipage} & \begin{minipage}[t]{0.25\columnwidth}\raggedright
\strut
\end{minipage} & \begin{minipage}[t]{0.22\columnwidth}\raggedright
read ()\strut
\end{minipage}\tabularnewline
\begin{minipage}[t]{0.19\columnwidth}\raggedright
dem\strut
\end{minipage} & \begin{minipage}[t]{0.22\columnwidth}\raggedright
\strut
\end{minipage} & \begin{minipage}[t]{0.25\columnwidth}\raggedright
\strut
\end{minipage} & \begin{minipage}[t]{0.22\columnwidth}\raggedright
none (looks like just encrypted cnd)\strut
\end{minipage}\tabularnewline
\begin{minipage}[t]{0.19\columnwidth}\raggedright
dsb\strut
\end{minipage} & \begin{minipage}[t]{0.22\columnwidth}\raggedright
YES\strut
\end{minipage} & \begin{minipage}[t]{0.25\columnwidth}\raggedright
\strut
\end{minipage} & \begin{minipage}[t]{0.22\columnwidth}\raggedright
read (unknown how well) (stitch data looks same as 10o)\strut
\end{minipage}\tabularnewline
\begin{minipage}[t]{0.19\columnwidth}\raggedright
dst\strut
\end{minipage} & \begin{minipage}[t]{0.22\columnwidth}\raggedright
YES\strut
\end{minipage} & \begin{minipage}[t]{0.25\columnwidth}\raggedright
\strut
\end{minipage} & \begin{minipage}[t]{0.22\columnwidth}\raggedright
read (complete) / write(unknown)\strut
\end{minipage}\tabularnewline
\begin{minipage}[t]{0.19\columnwidth}\raggedright
dsz\strut
\end{minipage} & \begin{minipage}[t]{0.22\columnwidth}\raggedright
YES\strut
\end{minipage} & \begin{minipage}[t]{0.25\columnwidth}\raggedright
\strut
\end{minipage} & \begin{minipage}[t]{0.22\columnwidth}\raggedright
read (unknown)\strut
\end{minipage}\tabularnewline
\begin{minipage}[t]{0.19\columnwidth}\raggedright
dxf\strut
\end{minipage} & \begin{minipage}[t]{0.22\columnwidth}\raggedright
\strut
\end{minipage} & \begin{minipage}[t]{0.25\columnwidth}\raggedright
\strut
\end{minipage} & \begin{minipage}[t]{0.22\columnwidth}\raggedright
read (Port to C. needs refactored)\strut
\end{minipage}\tabularnewline
\begin{minipage}[t]{0.19\columnwidth}\raggedright
edr\strut
\end{minipage} & \begin{minipage}[t]{0.22\columnwidth}\raggedright
\strut
\end{minipage} & \begin{minipage}[t]{0.25\columnwidth}\raggedright
\strut
\end{minipage} & \begin{minipage}[t]{0.22\columnwidth}\raggedright
read (C version is broken) / write (complete)\strut
\end{minipage}\tabularnewline
\begin{minipage}[t]{0.19\columnwidth}\raggedright
emd\strut
\end{minipage} & \begin{minipage}[t]{0.22\columnwidth}\raggedright
\strut
\end{minipage} & \begin{minipage}[t]{0.25\columnwidth}\raggedright
\strut
\end{minipage} & \begin{minipage}[t]{0.22\columnwidth}\raggedright
read (unknown)\strut
\end{minipage}\tabularnewline
\begin{minipage}[t]{0.19\columnwidth}\raggedright
exp\strut
\end{minipage} & \begin{minipage}[t]{0.22\columnwidth}\raggedright
YES\strut
\end{minipage} & \begin{minipage}[t]{0.25\columnwidth}\raggedright
\strut
\end{minipage} & \begin{minipage}[t]{0.22\columnwidth}\raggedright
read (unknown) / write(unknown)\strut
\end{minipage}\tabularnewline
\begin{minipage}[t]{0.19\columnwidth}\raggedright
exy\strut
\end{minipage} & \begin{minipage}[t]{0.22\columnwidth}\raggedright
YES\strut
\end{minipage} & \begin{minipage}[t]{0.25\columnwidth}\raggedright
\strut
\end{minipage} & \begin{minipage}[t]{0.22\columnwidth}\raggedright
read (need to fix external color loading)\strut
\end{minipage}\tabularnewline
\begin{minipage}[t]{0.19\columnwidth}\raggedright
fxy\strut
\end{minipage} & \begin{minipage}[t]{0.22\columnwidth}\raggedright
YES\strut
\end{minipage} & \begin{minipage}[t]{0.25\columnwidth}\raggedright
\strut
\end{minipage} & \begin{minipage}[t]{0.22\columnwidth}\raggedright
read (need to fix external color loading)\strut
\end{minipage}\tabularnewline
\begin{minipage}[t]{0.19\columnwidth}\raggedright
gnc\strut
\end{minipage} & \begin{minipage}[t]{0.22\columnwidth}\raggedright
\strut
\end{minipage} & \begin{minipage}[t]{0.25\columnwidth}\raggedright
\strut
\end{minipage} & \begin{minipage}[t]{0.22\columnwidth}\raggedright
none\strut
\end{minipage}\tabularnewline
\begin{minipage}[t]{0.19\columnwidth}\raggedright
gt\strut
\end{minipage} & \begin{minipage}[t]{0.22\columnwidth}\raggedright
\strut
\end{minipage} & \begin{minipage}[t]{0.25\columnwidth}\raggedright
\strut
\end{minipage} & \begin{minipage}[t]{0.22\columnwidth}\raggedright
read (need to fix external color loading)\strut
\end{minipage}\tabularnewline
\begin{minipage}[t]{0.19\columnwidth}\raggedright
hus\strut
\end{minipage} & \begin{minipage}[t]{0.22\columnwidth}\raggedright
YES\strut
\end{minipage} & \begin{minipage}[t]{0.25\columnwidth}\raggedright
\strut
\end{minipage} & \begin{minipage}[t]{0.22\columnwidth}\raggedright
read (unknown) / write (C version is broken)\strut
\end{minipage}\tabularnewline
\begin{minipage}[t]{0.19\columnwidth}\raggedright
inb\strut
\end{minipage} & \begin{minipage}[t]{0.22\columnwidth}\raggedright
YES\strut
\end{minipage} & \begin{minipage}[t]{0.25\columnwidth}\raggedright
\strut
\end{minipage} & \begin{minipage}[t]{0.22\columnwidth}\raggedright
read (buggy?)\strut
\end{minipage}\tabularnewline
\begin{minipage}[t]{0.19\columnwidth}\raggedright
jef\strut
\end{minipage} & \begin{minipage}[t]{0.22\columnwidth}\raggedright
YES\strut
\end{minipage} & \begin{minipage}[t]{0.25\columnwidth}\raggedright
\strut
\end{minipage} & \begin{minipage}[t]{0.22\columnwidth}\raggedright
write (need to fix the offsets when it is moving to another spot)\strut
\end{minipage}\tabularnewline
\begin{minipage}[t]{0.19\columnwidth}\raggedright
ksm\strut
\end{minipage} & \begin{minipage}[t]{0.22\columnwidth}\raggedright
YES\strut
\end{minipage} & \begin{minipage}[t]{0.25\columnwidth}\raggedright
\strut
\end{minipage} & \begin{minipage}[t]{0.22\columnwidth}\raggedright
read (unknown) / write (unknown)\strut
\end{minipage}\tabularnewline
\begin{minipage}[t]{0.19\columnwidth}\raggedright
pcd\strut
\end{minipage} & \begin{minipage}[t]{0.22\columnwidth}\raggedright
\strut
\end{minipage} & \begin{minipage}[t]{0.25\columnwidth}\raggedright
\strut
\end{minipage} & \begin{minipage}[t]{0.22\columnwidth}\raggedright
\strut
\end{minipage}\tabularnewline
\begin{minipage}[t]{0.19\columnwidth}\raggedright
pcm\strut
\end{minipage} & \begin{minipage}[t]{0.22\columnwidth}\raggedright
\strut
\end{minipage} & \begin{minipage}[t]{0.25\columnwidth}\raggedright
\strut
\end{minipage} & \begin{minipage}[t]{0.22\columnwidth}\raggedright
\strut
\end{minipage}\tabularnewline
\begin{minipage}[t]{0.19\columnwidth}\raggedright
pcq\strut
\end{minipage} & \begin{minipage}[t]{0.22\columnwidth}\raggedright
\strut
\end{minipage} & \begin{minipage}[t]{0.25\columnwidth}\raggedright
\strut
\end{minipage} & \begin{minipage}[t]{0.22\columnwidth}\raggedright
read (Port to C)\strut
\end{minipage}\tabularnewline
\begin{minipage}[t]{0.19\columnwidth}\raggedright
pcs\strut
\end{minipage} & \begin{minipage}[t]{0.22\columnwidth}\raggedright
BUGGY\strut
\end{minipage} & \begin{minipage}[t]{0.25\columnwidth}\raggedright
\strut
\end{minipage} & \begin{minipage}[t]{0.22\columnwidth}\raggedright
read (buggy / colors are not correct / after reading, writing any other
format is messed up)\strut
\end{minipage}\tabularnewline
\begin{minipage}[t]{0.19\columnwidth}\raggedright
pec\strut
\end{minipage} & \begin{minipage}[t]{0.22\columnwidth}\raggedright
\strut
\end{minipage} & \begin{minipage}[t]{0.25\columnwidth}\raggedright
\strut
\end{minipage} & \begin{minipage}[t]{0.22\columnwidth}\raggedright
read / write (without embedded images, sometimes overlooks some stitches
leaving a gap)\strut
\end{minipage}\tabularnewline
\begin{minipage}[t]{0.19\columnwidth}\raggedright
pel\strut
\end{minipage} & \begin{minipage}[t]{0.22\columnwidth}\raggedright
\strut
\end{minipage} & \begin{minipage}[t]{0.25\columnwidth}\raggedright
\strut
\end{minipage} & \begin{minipage}[t]{0.22\columnwidth}\raggedright
none\strut
\end{minipage}\tabularnewline
\begin{minipage}[t]{0.19\columnwidth}\raggedright
pem\strut
\end{minipage} & \begin{minipage}[t]{0.22\columnwidth}\raggedright
\strut
\end{minipage} & \begin{minipage}[t]{0.25\columnwidth}\raggedright
\strut
\end{minipage} & \begin{minipage}[t]{0.22\columnwidth}\raggedright
none\strut
\end{minipage}\tabularnewline
\begin{minipage}[t]{0.19\columnwidth}\raggedright
pes\strut
\end{minipage} & \begin{minipage}[t]{0.22\columnwidth}\raggedright
YES\strut
\end{minipage} & \begin{minipage}[t]{0.25\columnwidth}\raggedright
\strut
\end{minipage} & \begin{minipage}[t]{0.22\columnwidth}\raggedright
\strut
\end{minipage}\tabularnewline
\begin{minipage}[t]{0.19\columnwidth}\raggedright
phb\strut
\end{minipage} & \begin{minipage}[t]{0.22\columnwidth}\raggedright
\strut
\end{minipage} & \begin{minipage}[t]{0.25\columnwidth}\raggedright
\strut
\end{minipage} & \begin{minipage}[t]{0.22\columnwidth}\raggedright
\strut
\end{minipage}\tabularnewline
\begin{minipage}[t]{0.19\columnwidth}\raggedright
phc\strut
\end{minipage} & \begin{minipage}[t]{0.22\columnwidth}\raggedright
\strut
\end{minipage} & \begin{minipage}[t]{0.25\columnwidth}\raggedright
\strut
\end{minipage} & \begin{minipage}[t]{0.22\columnwidth}\raggedright
\strut
\end{minipage}\tabularnewline
\begin{minipage}[t]{0.19\columnwidth}\raggedright
rgb\strut
\end{minipage} & \begin{minipage}[t]{0.22\columnwidth}\raggedright
\strut
\end{minipage} & \begin{minipage}[t]{0.25\columnwidth}\raggedright
\strut
\end{minipage} & \begin{minipage}[t]{0.22\columnwidth}\raggedright
\strut
\end{minipage}\tabularnewline
\begin{minipage}[t]{0.19\columnwidth}\raggedright
sew\strut
\end{minipage} & \begin{minipage}[t]{0.22\columnwidth}\raggedright
YES\strut
\end{minipage} & \begin{minipage}[t]{0.25\columnwidth}\raggedright
\strut
\end{minipage} & \begin{minipage}[t]{0.22\columnwidth}\raggedright
\strut
\end{minipage}\tabularnewline
\begin{minipage}[t]{0.19\columnwidth}\raggedright
shv\strut
\end{minipage} & \begin{minipage}[t]{0.22\columnwidth}\raggedright
\strut
\end{minipage} & \begin{minipage}[t]{0.25\columnwidth}\raggedright
\strut
\end{minipage} & \begin{minipage}[t]{0.22\columnwidth}\raggedright
read (C version is broken)\strut
\end{minipage}\tabularnewline
\begin{minipage}[t]{0.19\columnwidth}\raggedright
sst\strut
\end{minipage} & \begin{minipage}[t]{0.22\columnwidth}\raggedright
\strut
\end{minipage} & \begin{minipage}[t]{0.25\columnwidth}\raggedright
\strut
\end{minipage} & \begin{minipage}[t]{0.22\columnwidth}\raggedright
none\strut
\end{minipage}\tabularnewline
\begin{minipage}[t]{0.19\columnwidth}\raggedright
svg\strut
\end{minipage} & \begin{minipage}[t]{0.22\columnwidth}\raggedright
\strut
\end{minipage} & \begin{minipage}[t]{0.25\columnwidth}\raggedright
YES\strut
\end{minipage} & \begin{minipage}[t]{0.22\columnwidth}\raggedright
\strut
\end{minipage}\tabularnewline
\begin{minipage}[t]{0.19\columnwidth}\raggedright
tap\strut
\end{minipage} & \begin{minipage}[t]{0.22\columnwidth}\raggedright
YES\strut
\end{minipage} & \begin{minipage}[t]{0.25\columnwidth}\raggedright
\strut
\end{minipage} & \begin{minipage}[t]{0.22\columnwidth}\raggedright
read (unknown)\strut
\end{minipage}\tabularnewline
\begin{minipage}[t]{0.19\columnwidth}\raggedright
u01\strut
\end{minipage} & \begin{minipage}[t]{0.22\columnwidth}\raggedright
\strut
\end{minipage} & \begin{minipage}[t]{0.25\columnwidth}\raggedright
\strut
\end{minipage} & \begin{minipage}[t]{0.22\columnwidth}\raggedright
\strut
\end{minipage}\tabularnewline
\begin{minipage}[t]{0.19\columnwidth}\raggedright
vip\strut
\end{minipage} & \begin{minipage}[t]{0.22\columnwidth}\raggedright
YES\strut
\end{minipage} & \begin{minipage}[t]{0.25\columnwidth}\raggedright
\strut
\end{minipage} & \begin{minipage}[t]{0.22\columnwidth}\raggedright
\strut
\end{minipage}\tabularnewline
\begin{minipage}[t]{0.19\columnwidth}\raggedright
vp3\strut
\end{minipage} & \begin{minipage}[t]{0.22\columnwidth}\raggedright
YES\strut
\end{minipage} & \begin{minipage}[t]{0.25\columnwidth}\raggedright
\strut
\end{minipage} & \begin{minipage}[t]{0.22\columnwidth}\raggedright
\strut
\end{minipage}\tabularnewline
\begin{minipage}[t]{0.19\columnwidth}\raggedright
xxx\strut
\end{minipage} & \begin{minipage}[t]{0.22\columnwidth}\raggedright
YES\strut
\end{minipage} & \begin{minipage}[t]{0.25\columnwidth}\raggedright
\strut
\end{minipage} & \begin{minipage}[t]{0.22\columnwidth}\raggedright
\strut
\end{minipage}\tabularnewline
\begin{minipage}[t]{0.19\columnwidth}\raggedright
zsk\strut
\end{minipage} & \begin{minipage}[t]{0.22\columnwidth}\raggedright
\strut
\end{minipage} & \begin{minipage}[t]{0.25\columnwidth}\raggedright
\strut
\end{minipage} & \begin{minipage}[t]{0.22\columnwidth}\raggedright
read (complete)\strut
\end{minipage}\tabularnewline
\bottomrule
\end{longtable}

Support for Singer FHE, CHE (Compucon) formats?

\hypertarget{embroidermodder-project-coding-standards}{%
\section{Embroidermodder Project Coding
Standards}\label{embroidermodder-project-coding-standards}}

A basic set of guidelines to use when submitting code.

\hypertarget{naming-conventions}{%
\subsection{Naming Conventions}\label{naming-conventions}}

Name variables and functions intelligently to minimize the need for
comments. It should be immediately obvious what information it
represents. Short names such as x and y are fine when referring to
coordinates. Short names such as i and j are fine when doing loops.

Variable names should be ``camelCase'', starting with a lowercase word
followed by uppercase word(s). C++ Class Names should be ``CamelCase'',
using all uppercase word(s). C Functions that attempt to simulate
namespacing, should be ``nameSpace\_camelCase''.

All files and directories shall be lowercase and contain no spaces.

\hypertarget{code-style}{%
\subsection{Code Style}\label{code-style}}

Tabs should not be used when indenting. Setup your IDE or text editor to
use 4 spaces.

\hypertarget{braces}{%
\subsubsection{Braces}\label{braces}}

For functions: please put each brace on a new line.

\begin{verbatim}
void function_definition(int argument)
{

}
\end{verbatim}

For control statements: please put the first brace on the same line.

\begin{verbatim}
if (condition) {

}
\end{verbatim}

Use exceptions sparingly.

Do not use ternary operator (?:) in place of if/else.

Do not repeat a variable name that already occurs in an outer scope.

\hypertarget{version-control}{%
\subsection{Version Control}\label{version-control}}

Being an open source project, developers can grab the latest code at any
time and attempt to build it themselves. We try our best to ensure that
it will build smoothly at any time, although occasionally we do break
the build. In these instances, please provide a patch, pull request
which fixes the issue or open an issue and notify us of the problem, as
we may not be aware of it and we can build fine.

Try to group commits based on what they are related to:
features/bugs/comments/graphics/commands/etc\ldots{}

\hypertarget{comments}{%
\subsection{Comments}\label{comments}}

When writing code, sometimes there are items that we know can be
improved, incomplete or need special clarification. In these cases, use
the types of comments shown below. They are pretty standard and are
highlighted by many editors to make reviewing code easier. We also use
shell scripts to parse the code to find all of these occurrences so
someone wanting to go on a bug hunt will be able to easily see which
areas of the code need more love.

\begin{verbatim}
//C++ Style Comments
//TODO: This code clearly needs more work or further review.
//BUG: This code is definitely wrong. It needs fixed.
//HACK: This code shouldn't be written this way or I don't feel right about it. There may a better solution.
//WARNING: Think twice (or more times) before changing this code. I put this here for a good reason.
//NOTE: This comment is much more important than lesser comments.
\end{verbatim}

libembroidery is written in C and adheres to C89 standards. This means
that any C99 or C++ comments will show up as errors when compiling with
gcc. In any C code, you must use:

\begin{verbatim}
/* C Style Comments */
/* TODO: This code clearly needs more work or further review. */
/* BUG: This code is definitely wrong. It needs fixed. */
/* HACK: This code shouldn't be written this way or I don't feel right about it. There may a better solution */
/* WARNING: Think twice (or more times) before changing this code. I put this here for a good reason. */
/* NOTE: This comment is much more important than lesser comments. */
\end{verbatim}

\hypertarget{ideas-1}{%
\section{Ideas}\label{ideas-1}}

\hypertarget{why-this-document-1}{%
\subsection{Why this document}\label{why-this-document-1}}

I've been trying to make this document indirectly through the Github
issues page and the website we're building but I think a
straightforward, plain-text file needs to be the ultimate backup for
this. Then I can have a printout while I'm working on the project.

\hypertarget{issues-1}{%
\subsection{Issues}\label{issues-1}}

\hypertarget{fix-before-version-2-1}{%
\subsubsection{Fix before Version 2}\label{fix-before-version-2-1}}

So I've had a few pieces of web infrastructure fail me recently and I
think it's worth noting. An issue that affects us is an issue that can
effect people who use our software.

\begin{enumerate}
\def\labelenumi{\arabic{enumi}.}
\tightlist
\item
  Googletests require a web connection to update and they update on each
  compilation.
\item
  Downloading and installing Qt has been a pain for some users (46Gb on
  possibly slow connections). I think it was davieboy64?
\item
  The documentation is, well better in that it's housed in the main
  repository, but I'm not a fan of the ``write once build many''
  approach as it means trying to weigh up how 3 versions are going to
  render.
\item
  Github is giving me a server offline (500) error and is still giving a
  bad ping.
\item
  OpenGL rendering within the application. This will allow for Realistic
  Visualization - Bump Mapping/OpenGL/Gradients?
\item
  JSON configuration (Started, see
  \texttt{head\ -n\ 50\ src/mainwindow.cpp.}) Ok this is changing
  slightly. embroidermodder should boot from the command line regardless
  of whether it is or is not installed (this helps with testing and
  running on machines without root). Therefore, it can create an
  initiation file but it won't rely on its existence to boot: this is
  what we currently do with settings.ini.
\item
  Get undo history widget back (BUG).
\item
  Switch website to a CMake build.
\item
  Mac Bundle, .tar.gz and .zip source archive.
\item
  NSIS installer for Windows, Debian package, RPM package
\item
  GUI frontend for embroider features that aren't supported by
  embroidermodder: flag selector from a table
\item
  Update all formats without color to check for edr or rgb files.
\item
  EmbroideryFLOSS - Color picker that displays catalog numbers and names
\item
  Setting for reverse scrolling direction (for zoom, vertical pan)
\item
  Stitching simulation
\item
  User designed custom fill
\item
  Keyboard zooming, panning
\item
  Advanced printing
\item
  Libembroidery 1.0
\item
  Better integrated help: I don't think the help should backend to a
  html file somewhere on the user's system. A better system would be a
  custom widget within the program that's searchable.
\item
  New embroidermodder2.ico 16x16 logo that looks good at that scale.
\item
  saving dst, pes, jef
\item
  Settings dialog: notify when the user is switching tabs that the
  setting has been changed, adding apply button is what would make sense
  for this to happen.
\item
  Update language translations
\item
  Replace KDE4 thumbnailer.
\item
  Import raster image
\item
  Statistics from 1.0, needs histogram.
\item
  SNAP/ORTHO/POLAR
\item
  Cut/copy allow post-selection
\item
  Layout into config
\item
  Notify user of data loss if not saving to an object format.
\item
  Add which formats to work with to preferences.
\item
  Cannot open file with \# in the name when opening multiple files but
  works with opening a single file.
\item
  Closing settings dialog with the X in the window saves settings rather
  than discarding them.
\item
  Otto theme icons: units, render, selectors, what's this icon doesn't
  scale
\item
  Layer manager and Layer switcher dock widget
\item
  test that all formats read data in correct scale (format details
  should match other programs).
\item
  Custom filter bug -- doesn't save changes in some cases.
\end{enumerate}

CAD command review:

\begin{enumerate}
\def\labelenumi{\arabic{enumi}.}
\tightlist
\item
  scale
\item
  circle
\item
  offset
\item
  extend
\item
  trim
\item
  BreakAtPoint
\item
  Break2Points
\item
  Fillet
\item
  star
\item
  singlelinetext
\item
  Chamfer
\item
  split
\item
  area
\item
  time
\item
  pickadd
\item
  zoomfactor
\item
  product
\item
  program
\item
  zoomwindow
\item
  divide
\item
  find
\item
  record
\item
  playback
\item
  rotate
\item
  rgb
\item
  move
\item
  grid
\item
  griphot
\item
  gripcolor
\item
  gripcool
\item
  gripsize
\item
  highlight
\item
  units
\item
  locatepoint
\item
  distance
\item
  arc
\item
  ellipse
\item
  array
\item
  point
\item
  polyline
\item
  polygon
\item
  rectangle
\item
  line
\item
  arc (rt)
\item
  dolphin
\item
  heart
\end{enumerate}

So, it means weighing up some simplifications.

\hypertarget{fix-for-version-2.1-1}{%
\subsubsection{Fix for Version 2.1}\label{fix-for-version-2.1-1}}

\hypertarget{fix-eventually-1}{%
\subsubsection{Fix eventually}\label{fix-eventually-1}}

\hypertarget{googletests-1}{%
\subsubsection{googletests}\label{googletests-1}}

gtests are non-essential, testing is for developers not users so we can
choose our own framework. I think the in-built testing for libembroidery
was good and I want to re-instate it.

\hypertarget{qt-and-dependencies-1}{%
\subsubsection{Qt and dependencies}\label{qt-and-dependencies-1}}

I'm switching to SDL2 (which is a whole other conversation) which means
we can ship it with the source code package meaning only a basic build
environment is necessary to build it.

\hypertarget{documentation-1}{%
\subsubsection{Documentation}\label{documentation-1}}

Can we treat the website being a duplicate of the docs a non-starter?
I'd be happier with tex/pdf only and (I know this is counter-intuitive)
one per project.

\hypertarget{social-platform-1}{%
\subsubsection{Social Platform}\label{social-platform-1}}

So\ldots{} all the issues and project boards etc. being on Github is all
well and good assuming that we have our own copies. But we don't if
Github goes down or some other major player takes over the space and we
have to move (again, since this started on SourceForge).

This file is a backup for that which is why I'm repeating myself between
them.

\hypertarget{json-data-ideas-1}{%
\subsection{JSON data Ideas}\label{json-data-ideas-1}}

So:

\begin{enumerate}
\def\labelenumi{\arabic{enumi}.}
\tightlist
\item
  Port \texttt{settings.ini} to \texttt{settings.json}.
\item
  Place \texttt{settings.json} in \texttt{\$HOME/.embroidermodder} (or
  equivalent, see the homedir function in \texttt{gui.c}).
\item
  Parse JSON using cJSON (we have the new parseJSON function).
\item
  Better structure for settings data so parse and load JSON is easier
  and there's less junk in global variables. A structure similar to a
  Python dict that uses constants like the sketch below.
\end{enumerate}

\hypertarget{why-json-over-ini-1}{%
\subsubsection{Why JSON over ini?}\label{why-json-over-ini-1}}

\begin{enumerate}
\def\labelenumi{\arabic{enumi}.}
\tightlist
\item
  We need to hand-write \emph{a} system because the current system is Qt
  dependent anyway.
\item
  This way we can store more complex data structures in the same system
  including the layout of the widgets which may be user configured (see
  Blender and GIMP).
\item
  Also it's easier to share information formatted this way between
  systems because most systems us JSON or XML data: there's better
  support for converting complex data this way.
\end{enumerate}

\hypertarget{sketch-of-a-settings-system-1}{%
\subsubsection{Sketch of a settings
system}\label{sketch-of-a-settings-system-1}}

\begin{verbatim}
#define SETTING_interface_scale    16

...

char int_settings_labels[] = {
...
"interface scale" /* the sixteenth entry */
...
"%" /* terminator character */
};

...

    /* to use the setting */
    scale_interface(int_setting[SETTING_interface_scale]);

    /* to set setting */
    int_setting[SETTING_interface_scale] = 16;

    /* to make the JSON storage */
    for (i=0; int_settings_labels[i][0] != '%'; i++) {
        fprintf(setting_file, "\"%s\" :%d,\n", int_settings_labels[i], int_settings[i]);
    }
\end{verbatim}

This would all be in C, and wouldn't rely on Qt at all. We already use a
system like this in \texttt{libembroidery} so hopefully devs on both
would get the pattern.

\hypertarget{design-1}{%
\subsection{Design}\label{design-1}}

These are key bits of reasoning behind why the software is built the way
it is.

\hypertarget{scripting-overhaul-1}{%
\subsubsection{Scripting Overhaul}\label{scripting-overhaul-1}}

Originally Embroidermodder had a terminal widget, this is why we removed
it.

\begin{quote}
ROBIN: I think supporting scripting within Embroidermodder doesn't make
sense.

All features that use scripting can be part of libembroidery instead.
Users who are capable of using scripting won't need it, they can alter
their embroidery files in CSV format, or import pyembroidery to get
access. It makes maintaining the code a lot more complicated, especially
if we move away from Qt. Users who don't want the scripting feature will
likely be confused by it, since we say that's what libembroidery,
embroider and pyembroidery are for.

How about a simpler ``call user shell'' feature? Similar to texmaker we
just call system on a batch or shell script supplied by the user and it
processes the file directly then the software reloads the file. Then we
aren't parsing it directly.

I don't want to change this without Josh's support because it's a fairly
major change.

\begin{quote}
JOSH: I totally agree.

I like the idea of scripting just so people that know how to code could
write their own designs without needing to fully build the app.
Scripting would be a very advanced feature that most users would be
confused by. Libembroidery would be a good fit for advanced features.
\end{quote}
\end{quote}

\hypertarget{perennial-jobs-1}{%
\subsection{Perennial Jobs}\label{perennial-jobs-1}}

\begin{enumerate}
\def\labelenumi{\arabic{enumi}.}
\tightlist
\item
  Check for memory leaks
\item
  Clear compiler warnings on \texttt{-Wall\ -ansi\ -pedantic} for C.
\item
\end{enumerate}

\hypertarget{developing-for-android-1}{%
\subsubsection{Developing for Android}\label{developing-for-android-1}}

https://developer.android.com/studio/projects/add-native-code

\begin{verbatim}
apt install google-android-ndk-installer cmake lldb gradle
\end{verbatim}

\hypertarget{bibilography-1}{%
\subsection{Bibilography}\label{bibilography-1}}
